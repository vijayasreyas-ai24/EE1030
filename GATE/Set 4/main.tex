\let\negmedspace\undefined
\let\negthickspace\undefined
\documentclass[journal]{IEEEtran}
\usepackage[a5paper, margin=8mm, onecolumn]{geometry}
%\usepackage{lmodern} % Ensure lmodern is loaded for pdflatex
\usepackage{tfrupee} % Include tfrupee package

\setlength{\headheight}{1cm} % Set the height of the header box
\setlength{\headsep}{0mm}     % Set the distance between the header box and the top of the text

\usepackage{gvv-book}
\usepackage{gvv}
\usepackage{cite}
\usepackage{amsmath,amssymb,amsfonts,amsthm}
\usepackage{algorithmic}
\usepackage{graphicx}
\usepackage{textcomp}
\usepackage{xcolor}
\usepackage{txfonts}
\usepackage{listings}
\usepackage{enumitem}
\usepackage{mathtools}
\usepackage{gensymb}
\usepackage{comment}
\usepackage[breaklinks=true]{hyperref}
\usepackage{tkz-euclide} 
\usepackage{listings}
% \usepackage{gvv}                                        
\def\inputGnumericTable{}                                 
\usepackage[latin1]{inputenc}                                
\usepackage{color}                                            
\usepackage{array}                                            
\usepackage{longtable}                                       
\usepackage{calc}                                             
\usepackage{multirow}                                         
\usepackage{hhline}                                           
\usepackage{ifthen}                                           
\usepackage{lscape}
\begin{document}

\bibliographystyle{IEEEtran}
\vspace{3cm}

\title{2010 Mechanical Engineering}
\author{AI24BTECH11003 - Badde Vijaya Sreyas}
% \maketitle
% \newpage
% \bigskip
{\let\newpage\relax\maketitle}

\renewcommand{\thefigure}{\theenumi}
\renewcommand{\thetable}{\theenumi}
\setlength{\intextsep}{10pt} % Space between text and floats


\numberwithin{equation}{enumi}
\numberwithin{figure}{enumi}
\renewcommand{\thetable}{\theenumi}

\begin{enumerate}
\setcounter{enumi}{0}
 
%1
    \item The parabolic arc $y=\sqrt{x},1\leq x\leq2$  is revolved around the axis. The volume of the solid of revolution is

        \begin{multicols}{4}
            \begin{enumerate}
                \item $\frac{\pi}{4}$
                \item $\frac{\pi}{2}$
                \item $\frac{3\pi}{4}$
                \item $\frac{3\pi}{2}$
            \end{enumerate}
        \end{multicols}

%2
    \item The Blausius equation, $\frac{d^3f}{d\eta^3}+\frac{f}{2}\frac{d^2f}{d\eta^2}=0$, is a 


            \begin{enumerate}
                \item second order nonlinear ordinary differential equation
                \item third order nonlinear ordinary differential equation
                \item third order linear ordinary differential equation
                \item mixed order nonlinear ordinary differential equation
            \end{enumerate}


%3
    \item The value of the integral $\overset{\infty}{\underset{-\infty}{\int}}\frac{dx}{1+x^2}$ is

        \begin{multicols}{4}
            \begin{enumerate}
                \item $-\pi$
                \item $-\frac{\pi}{2}$
                \item $\frac{\pi}{2}$
                \item $\pi$
            \end{enumerate}
        \end{multicols}


%4
    \item The modulus of the complex number $\brak{\frac{3+4i}{1-2i}}$
    
		\begin{multicols}{4}
			\begin{enumerate}
				\item $5$
				\item $\sqrt{5}$
				\item $\frac{1}{\sqrt{5}}$
				\item $\frac{1}{5}$
			\end{enumerate}
		\end{multicols}

%5
    \item The function $y=\abs{2-3x}$

       \begin{enumerate}
            \item is continuous for all $\forall x\in R$ and differentiable $\forall x\in R$
            \item is continuous for all $\forall x\in R$ and differentiable $\forall x\in R$ except at $x=\frac{3}{2}$
            \item is continuous for all $\forall x\in R$ and differentiable $\forall x\in R$ except at $x=\frac{2}{3}$
            \item is continuous for all $\forall x\in R$ except at $x=3$ and differentiable $\forall x\in R$
        \end{enumerate}
  
%6
    \item Mobility of a statically indeterminate structure is

    \begin{multicols}{4}
        \begin{enumerate}
            \item $\leq -1$
            \item $0$
            \item $1$
            \item $\geq2$
        \end{enumerate}
    \end{multicols}

%7
    \item There are 2 points P and Q on a planar rigid body. The relative velocity between the two points 

        
            \begin{enumerate}
                \item should always be along PQ
                \item can be oriented along any direction
                \item should always be perpendicular to PQ
                \item should be along QP when the body undergoes pure translation
            \end{enumerate}
        
		
%8
    \item The state of plane-stress at a point is given by $\sigma_x--200MPa,\sigma_y=100MPa$, and $\tau_{xy}=100MPa$. The maximum shear stress in MPa is

        \begin{multicols}{4}
            \begin{enumerate}
                \item 111.8
                \item 150.1
                \item 180.3
                \item 223.6
            \end{enumerate}
        \end{multicols}

%9
    \item Which of the following statements is INCORRECT?


            \begin{enumerate}
                \item Grashof's rule states that for a planar crank-rocker four bar mechanism, the sum of the shortest and longest link lengths cannot be less than the sum of the remaining two link lengths.
                \item Inversions of a mechanism are created by fixing different links one at a time.
                \item Geneva mechanism is an intermittent motion device
                \item Gruebler's criterion assumes mobility of a planar mechanism to be one.
            \end{enumerate}


%10
    \item The natural frequency of a spring-mass system on earth is $\omega_n$. The natural frequency of this system on the moon $\brak{g_{moon}=\frac{g_{earth}}{6}}$ is

        \begin{multicols}{4}
            \begin{enumerate}
                \item $\omega_n$
                \item $0.408\omega_n$
                \item $0.204\omega_n$
                \item $0.167\omega_n$
            \end{enumerate}
        \end{multicols}
        
%11
    \item Tooth interference in an external involute spur gear pair can be reduced by

        
            \begin{enumerate}
                \item decreasing center distance between gear pair
                \item decreasing module
                \item decreasing pressure angle
                \item increasing number of gear teeth
            \end{enumerate}
        

%12
    \item For the stability of a floating body, under the influence of gravity alone, which of the following is TRUE?

        
            \begin{enumerate}
                \item Metacentre should be below centre of gravity
                \item Metacentre should be above centre of gravity
                \item Metacentre and centre of gravity must lie on the same horizontal line
                \item Metacentre and centre of gravity must lie on the same vertical line
            \end{enumerate}
        
        
%13
    \item The maximum velocity of a one-dimensional incompressible fully developed viscous flow, between two fixed parallel plates, is $6\frac{m}{s}$ . The mean velocity (in $\frac{m}{s}$) of the flow is

        \begin{multicols}{4}
            \begin{enumerate}
                \item $2$
                \item $3$
                \item $4$
                \item $5$
            \end{enumerate}
        \end{multicols}

\end{enumerate}

\end{document}
