%iffalse
\let\negmedspace\undefined
\let\negthickspace\undefined
\documentclass[journal,12pt,twocolumn]{IEEEtran}
\usepackage{cite}
\usepackage{amsmath,amssymb,amsfonts,amsthm}
\usepackage{algorithmic}
\usepackage{graphicx}
\usepackage{textcomp}
\usepackage{xcolor}
\usepackage{txfonts}
\usepackage{listings}
\usepackage{enumitem}
\usepackage{mathtools}
\usepackage{gensymb}
\usepackage{comment}
\usepackage[breaklinks=true]{hyperref}
\usepackage{tkz-euclide} 
\usepackage{listings}
\usepackage{gvv}                                        
%\def\inputGnumericTable{}                                 
\usepackage[latin1]{inputenc}                                
\usepackage{color}                                            
\usepackage{array}                                            
\usepackage{longtable}                                       
\usepackage{calc}                                             
\usepackage{multirow}                                         
\usepackage{hhline}                                           
\usepackage{ifthen}                                           
\usepackage{lscape}
\usepackage{tabularx}
\usepackage{array}
\usepackage{float}
\usepackage{multicol}


\newtheorem{theorem}{Theorem}[section]
\newtheorem{problem}{Problem}
\newtheorem{proposition}{Proposition}[section]
\newtheorem{lemma}{Lemma}[section]
\newtheorem{corollary}[theorem]{Corollary}
\newtheorem{example}{Example}[section]
\newtheorem{definition}[problem]{Definition}
\newcommand{\BEQA}{\begin{eqnarray}}
\newcommand{\EEQA}{\end{eqnarray}}
\newcommand{\define}{\stackrel{\triangle}{=}}
\theoremstyle{remark}
\newtheorem{rem}{Remark}

% Marks the beginning of the document
\begin{document}
\bibliographystyle{IEEEtran}
\vspace{3cm}

\title{Assignment - 1}
\author{AI24BTECH11003 - B. Vijaya Sreyas}
\maketitle
\newpage
\bigskip

\renewcommand{\thefigure}{\theenumi}
\renewcommand{\thetable}{\theenumi}

\large\section{\textbf{17.Indefinite Integrals - Section B}}\small

\begin{enumerate}[label=\textcolor{magenta}{\arabic*.}]
		\setcounter{enumi}{4}
%1
	\item The value of $\sqrt{2}\int\frac{sinxdx}{sin(x-\frac{\pi}{4})}$
		\hfill{\textcolor{magenta}{[2008]}}

		(a)$x+log|cos(x-\frac{\pi}{4})|+c$

		(b)$x-log|sin(x-\frac{\pi}{4})|+c$

		(c)$x+log|sin(x-\frac{\pi}{4})|+c$

		(d)$x-log|cos(x-\frac{\pi}{4})|+c$

%2
	\item If the $\int\frac{5tanx}{tanx-2}dx=x+aln|sinx-2cosx|+k$, then $a$ is equal to
		\hfill{\textcolor{magenta}{[2012]}}

		\begin{multicols}{4}
			\begin{enumerate}[label=(\alph*)]
				\item -1
				\item 2
				\item 1
				\item 2
			\end{enumerate}
		\end{multicols}
		
%3
	\item If $\int f(x)dx=\psi(x)$, then $\int x^5f(x^3)dx$ is equal to:

		\hfill{\textcolor{magenta}{[JEE M 2013]}}

		(a) $\frac{1}{3}[x^3\psi(x^3)-\int x^2\psi(x^3)dx] + C$

		(b) $\frac{1}{3}x^3\psi(x^3)-3\int x^3\psi(x^3)dx + C$

		(c) $\frac{1}{3}x^3\psi(x^3)-\int x^2\psi(x^3)dx + C$

		(d) $\frac{1}{3}[x^3\psi(x^3)-\int x^3\psi(x^3)dx] + C$

%4
	\item The integral $\int(1+x-\frac{1}{x})e^{x+\frac{1}{x}}dx$ is equal to

		\hfill{\textcolor{magenta}{[JEE M 2014]}}

		\begin{multicols}{2}
			\begin{enumerate}[label=(\alph*)]
				\item $(x+1)e^{x+\frac{1}{x}}+c$
				\item $-xe^{x+\frac{1}{x}}+c$
				\item $(x-1)e^{x+\frac{1}{x}}+c$
				\item $xe^{x+\frac{1}{x}}+c$
			\end{enumerate}
		\end{multicols}

%5
	\item The integral $\int\frac{dx}{x^2(x^4+1)^{3/4}}$ equals:
		\hfill{\textcolor{magenta}{[JEE M 2015]}}

		\begin{multicols}{2}
			\begin{enumerate}[label=(\alph*)]
				\item $-(x^4+1)^{\frac{1}{4}}+c$
				\item $-(\frac{x^4+1}{x^4})+c$
				\item $(\frac{x^4+1}{x^4})^{\frac{1}{4}}+c$
				\item $(x^4+1)^{\frac{1}{4}}+c$
			\end{enumerate}
		\end{multicols}

%6
	\item The integral $\int\frac{2x^{12}+5x^9}{(x^5+x^3+1)^3}dx$ is equal to
		\hfill{\textcolor{magenta}{[JEE M 2016]}}

		\begin{multicols}{2}
			\begin{enumerate}[label=(\alph*)]
				\item $\frac{x^5}{2(x^5-x^3+1)^2}+C$
				\item $\frac{-x^{10}}{2(x^5+x^3+10^2}+C$
				\item $\frac{-x^5}{(x^5+x^3+1)^2}+C$
				\item $\frac{x^{10}}{2(x^5+x^3+1)}+C$
			\end{enumerate}
		\end{multicols}
		where C is an arbitrary constant

%7
	\item Let I$_n$ = $\int$tan$^x$dx, (n$>$1). I$_4$+I$_6$ = a tan$^5$x + bx$^5$ + C, where C is constant of integration, then the ordered pair (a, b) is equal to :
		\hfill{\textcolor{magenta}{[JEE M 2017]}}

		\begin{multicols}{4}
			\begin{enumerate}[label=(\alph*)]
				\item ($-\frac{1}{5}$, 0)
				\item ($-\frac{1}{5}$, 1)
				\item ($\frac{1}{5}$, 0)
				\item ($\frac{1}{5}$,-1)
			\end{enumerate}
		\end{multicols}
		
%8
	\item The integral $\int\frac{sin^2xcos^2x}{(sin^5x+cos^3xsin^2x+sin^3xcos^2x+cos^5x)^2}$dx is equal to
		\hfill{\textcolor{magenta}{[JEE M 2018]}}
		
		\begin{multicols}{2}
			\begin{enumerate}[label=(\alph*)]
				\item $\frac{-1}{3(1+tan^3x)}$ + C
				\item $\frac{1}{1+cot^3x}$ + C
				\item $\frac{-1}{1+cot^3x}$ + C
				\item $\frac{1}{3(1+tan^3x)}$ + C
			\end{enumerate}
		\end{multicols}
		
%9
	\item For $x^2\neq$ n$\pi+1$, n$\in$N (the set of natural numbers), the integral $\int x\sqrt{\frac{2sin(x^2-1)-sin2(x^2-1)}{2sin(x^2-1)+sin2(x^2-1)}}$dx is equal to:

		\hfill{\textcolor{magenta}{[JEE M 2019 - 9 Jan(M)]}}

		\begin{multicols}{1}
			\begin{enumerate}[label=(\alph*)]
				\item log$_e|\frac{1}{2}sec^2(x^2-1)|$ + c
				\item $\frac{1}{2}$log$_e|sec^2(\frac{x^2-1}{2})|$ + c
				\item $\frac{1}{2}$log$_e|sec^2(\frac{x^2-1}{2})|$ + c
				\item log$_2|sec(\frac{x^2-1}{2})|$ + c
			\end{enumerate}
		\end{multicols}
		(where c is a constant of integration)

%10
	\item The integral $\int sec^{2/3}xcosec^{4/3}x dx$ is equal to

		\hfill{\textcolor{magenta}{[JEE M 2019 - 9 April (M)]}}

		\begin{multicols}{2}
			\begin{enumerate}[label=(\alph*)]
				\item -3tan$^{-1/3}$x+C
				\item -$\frac{3}{4}$tan$^{-4/3}$x+C
				\item -3cot$^{-1/3}$x+C
				\item 3tan$^{-1/3}$+C
			\end{enumerate}
		\end{multicols}
		(Here, C is a constant of integration)
\end{enumerate}

\large\section{\textbf{18. Definite Integrals - Section B}}\small
\begin{enumerate}[label=\textcolor{magenta}{\arabic*.}]
		\setcounter{enumi}{30}

%31

	\item The area of the region bounded by the parabola $(y-2)^2=x-1$, the tangent of the parabola at the point (2, 3) and the $x$-axis is:
		\hfill{\textcolor{magenta}{[2009]}}

		\begin{multicols}{4}
			\begin{enumerate}[label=(\alph*)]
				\item 6
				\item 9
				\item 12
				\item 3
			\end{enumerate}
		\end{multicols}
			
%32	
		
	\item $\int_0^\pi[cotx]dx$, where [.] denotes the greatest integer function, is equal to
		\hfill{\textcolor{magenta}{[2009]}}

		\begin{multicols}{4}
			\begin{enumerate}[label=(\alph*)]
				\item 1
				\item -1
				\item $-\frac{\pi}{2}$
				\item $\frac{\pi}{2}$
			\end{enumerate}
		\end{multicols}

%33

	\item The area bounded between the curves $y=cosx$ and $y=sinx$ between the ordinates $x=0$ and $x=\frac{3\pi}{2}$ is

		\hfill{\textcolor{magenta}{[2010]}}

		\begin{multicols}{2}
			\begin{enumerate}[label=(\alph*)]
				\item $4\sqrt{2}+2$
				\item $4\sqrt{2}-1$
				\item $4\sqrt{2}+1$
				\item $4\sqrt{2}-2$
			\end{enumerate}
		\end{multicols}

%34

	\item Let $p(x)$ be a function defined on \textbf{R} such that $p'(x)=p'(1-x)$, for all $x\in[0,1],p(0)=1$ and $p(1)=41$. Then $\int_0^1p(x)dx$ equals
		\hfill{\textcolor{magenta}{[2010]}}

					\begin{multicols}{4}
			\begin{enumerate}[label=(\alph*)]
				\item 21
				\item 41
				\item 42
				\item $\sqrt{41}$
			\end{enumerate}
		\end{multicols}

%35

	\item The value of $\int_0^1\frac{8log(1+x)}{1+x^2}dx$ is
		\hfill{\textcolor{magenta}{[2011]}}

		\begin{multicols}{2}
			\begin{enumerate}[label=(\alph*)]
				\item $\frac{\pi}{8}$log2
				\item $\frac{\pi}{2}$log2
				\item log 2
				\item $\pi$ log2
			\end{enumerate}
		\end{multicols}

%36

	\item The area of the region enclosed by the curves $y=x, x=e, y=\frac{1}{x}$ and the positive x axis is
		\hfill{\textcolor{magenta}{[2011]}}

		\begin{multicols}{2}
			\begin{enumerate}[label=(\alph*)]
				\item 1 square unit
				\item $\frac{3}{2}$ square units
				\item $\frac{5}{2}$ square units
				\item $\frac{1}{2}$ square unit
			\end{enumerate}
		\end{multicols}

%37

	\item The area between the parabolas:$x^2=\frac{y}{4}$ and $x^2=9y$ and the straight line $y=2$ is:
		\hfill{\textcolor{magenta}{[2012]}}

		\begin{multicols}{4}
			\begin{enumerate}[label=(\alph*)]
				\item $20\sqrt{2}$
				\item $\frac{10\sqrt{2}}{3}$
				\item $\frac{20\sqrt{2}}{3}$
				\item $10\sqrt{2}$
			\end{enumerate}
		\end{multicols}

%38

	\item If $g(x)=\int_0^xcos4t dt$, then $g(x+\pi)$ equals
		\hfill{\textcolor{magenta}{[2012]}}

		\begin{multicols}{2}
			\begin{enumerate}[label=(\alph*)]
				\item $\frac{g(x)}{g(\pi)}$
				\item $g(x)+g(\pi)$
				\item $g(x)-g(\pi)$
				\item $g(x).g(\pi)$
			\end{enumerate}
		\end{multicols}

%39

	\item \textbf{Statement-1 :} The value of the integral $\int_{\pi/6}^{\pi/3}\frac{dx}{1+\sqrt{tanx}}$ is equal to $\pi/6$

		\textbf{Statement-2 :} $\int_a^bf(x)dx=\int_a^bf(a+b-x)dx$.

		\hfill{\textcolor{magenta}{[JEE M 2013]}}
		
		(a) Statement-1 is true; Statement-2 is true; Statement-2 is a correct explanation for Statement-1

		(b) Statement-1 is true; Statement-2 is true; Statement-2 is not a correct explanation fo Statement-1

		(c) Statement-1 is true; Statement-2 is false

		(d) Statement-1 is false; Statement-2 is true

%40

	\item The area (in square units) bounded by the curves $y=\sqrt{x}$, $2y-x+3=0$, $x$-axis, and lying in the first quadrant is :
		\hfill{\textcolor{magenta}{[JEE M 2013]}}

		\begin{multicols}{4}
			\begin{enumerate}[label=(\alph*)]
				\item 9
				\item 36
				\item 18
				\item $\frac{27}{4}$
			\end{enumerate}
		\end{multicols}


%41

	\item The integral $\int_0^{\pi}\sqrt{1+4sin^2\frac{x}{2}-4sin\frac{x}{2}}dx$ equals:

		\hfill{\textcolor{magenta}{[JEE M 2014]}}

		\begin{multicols}{2}
			\begin{enumerate}[label=(\alph*)]
				\item $4\sqrt{3}-4$
				\item $4\sqrt{3}-4-\frac{\pi}{3}$
				\item $\pi-4$
				\item $\frac{2\pi}{3}-4-4\sqrt{3}$
			\end{enumerate}
		\end{multicols}

%42

	\item The area of the region described by $A=\{(x,y):x^2+y^2\leq1$ and $y^2\leq1-x\}$ is:
		\hfill{\textcolor{magenta}{[JEE M 2014]}}

		\begin{multicols}{4}
			\begin{enumerate}[label=(\alph*)]
				\item $\frac{\pi}{2}-\frac{2}{3}$
				\item $\frac{\pi}{2}+\frac{2}{3}$
				\item $\frac{\pi}{2}+\frac{4}{3}$
				\item $\frac{\pi}{2}-\frac{4}{3}$
			\end{enumerate}
		\end{multicols}

%43

	\item The area (in sq. units) of the region described by $\{(x,y):y^2\leq2x$ and $y\geq4x-1\}$ is
		\hfill{\textcolor{magenta}{[JEE M 2015]}}

		\begin{multicols}{4}
			\begin{enumerate}[label=(\alph*)]
				\item $\frac{15}{64}$
				\item $\frac{9}{32}$
				\item $\frac{7}{32}$
				\item $\frac{5}{64}$
			\end{enumerate}
		\end{multicols}

%44

	\item The integral $\int_2^4\frac{log x^2}{log x^2 + log(36-12x+x^2)}dx$ is equal to:

		\hfill{\textcolor{magenta}{[JEE M 2015]}}

		\begin{multicols}{4}
			\begin{enumerate}[label=(\alph*)]
				\item 1
				\item 6
				\item 2
				\item 4
			\end{enumerate}
		\end{multicols}

%45

	\item The area (in sq. units) of the region $\{(x,y):y^2\geq2x$ and $x^2+y^2\leq4x, x\geq0, y\geq0\}$ is
		\hfill{\textcolor{magenta}{[JEE M 2016]}}

		\begin{multicols}{2}
			\begin{enumerate}[label=(\alph*)]
				\item $\pi-\frac{4\sqrt{2}}{3}$
				\item $\frac{\pi}{2}-\frac{2\sqrt{2}}{3}$
				\item $\pi-\frac{4}{3}$
				\item $\pi-\frac{8}{3}$ 
			\end{enumerate}
		\end{multicols} 
\end{enumerate}
\end{document}
