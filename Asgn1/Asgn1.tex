%iffalse
\let\negmedspace\undefined
\let\negthickspace\undefined
\documentclass[journal,12pt,twocolumn]{IEEEtran}
\usepackage{cite}
\usepackage{amsmath,amssymb,amsfonts,amsthm}
\usepackage{algorithmic}
\usepackage{graphicx}
\usepackage{textcomp}
\usepackage{xcolor}
\usepackage{txfonts}
\usepackage{listings}
\usepackage{enumitem}
\usepackage{mathtools}
\usepackage{gensymb}
\usepackage{comment}
\usepackage[breaklinks=true]{hyperref}
\usepackage{tkz-euclide} 
\usepackage{listings}
\usepackage{gvv}                                        
%\def\inputGnumericTable{}                                 
\usepackage[latin1]{inputenc}                                
\usepackage{color}                                            
\usepackage{array}                                            
\usepackage{longtable}                                       
\usepackage{calc}                                             
\usepackage{multirow}                                         
\usepackage{hhline}                                           
\usepackage{ifthen}                                           
\usepackage{lscape}
\usepackage{tabularx}
\usepackage{array}
\usepackage{float}
\usepackage{multicol}


\newtheorem{theorem}{Theorem}[section]
\newtheorem{problem}{Problem}
\newtheorem{proposition}{Proposition}[section]
\newtheorem{lemma}{Lemma}[section]
\newtheorem{corollary}[theorem]{Corollary}
\newtheorem{example}{Example}[section]
\newtheorem{definition}[problem]{Definition}
\newcommand{\BEQA}{\begin{eqnarray}}
\newcommand{\EEQA}{\end{eqnarray}}
\newcommand{\define}{\stackrel{\triangle}{=}}
\theoremstyle{remark}
\newtheorem{rem}{Remark}

% Marks the beginning of the document
\begin{document}
\bibliographystyle{IEEEtran}
\vspace{3cm}

\title{Assignment - 1}
\author{AI24BTECH11003 - B. Vijaya Sreyas}
\maketitle
\newpage
\bigskip

\renewcommand{\thefigure}{\theenumi}
\renewcommand{\thetable}{\theenumi}

17.Indefinite Integrals - Section B

5)
	 The value of $\sqrt{2}\int\frac{\sin xdx}{\sin \brak{x-\frac{\pi}{4}}}$

		\hfill{(2008)}

		(a)$x+\log\abs{\cos \brak{x-\frac{\pi}{4}}}+c$

		(b)$x-\log\abs{\sin \brak{x-\frac{\pi}{4}}}+c$

		(c)$x+\log\abs{\sin \brak{x-\frac{\pi}{4}}}+c$

		(d)$x-\log\abs{\cos \brak{x-\frac{\pi}{4}}}+c$

6)
	 If the $\int\frac{5\tan x}{\tan x-2}dx=x+a\ln \abs{\sin x-2\cos x}+k$, then $a$ is equal to

		\hfill{(2018)}

		\begin{multicols}{4}
			\begin{enumerate}
				\item -1
				\item 2
				\item 1
				\item 2
			\end{enumerate}
		\end{multicols}
		
7)
	If $\int f\brak{x}dx=\psi\brak{x}$, then $\int x^5f\brak{x^3}dx$ is equal to:

		\hfill{(JEE M 2013)}

		(a) $\frac{1}{3}\sbrak{x^3\psi\brak{x^3}-\int x^2\psi(x^3)dx} + C$

		(b) $\frac{1}{3}x^3\psi\brak{x^3}-3\int x^3\psi(x^3)dx + C$

		(c) $\frac{1}{3}x^3\psi\brak{x^3}-\int x^2\psi(x^3)dx + C$

		(d) $\frac{1}{3}\sbrak{x^3\psi\brak{x^3}-\int x^3\psi(x^3)dx} + C$

8)
	 The integral $\int\brak{1+x-\frac{1}{x}}e^{x+\frac{1}{x}}dx$ is equal to

		\hfill{(JEE M 2014)}

		\begin{multicols}{2}
			\begin{enumerate}
				\item $\brak{x+1}e^{x+\frac{1}{x}}+c$
				\item $-xe^{x+\frac{1}{x}}+c$
				\item $\brak{x-1}e^{x+\frac{1}{x}}+c$
				\item $xe^{x+\frac{1}{x}}+c$
			\end{enumerate}
		\end{multicols}

9)
	 The integral $\int\frac{dx}{x^2(x^4+1)^{3/4}}$ equals:

		\hfill{(JEE M 2015)}

		\begin{multicols}{2}
			\begin{enumerate}
				\item $-\brak{x^4+1}^{\frac{1}{4}}+c$
				\item $-\brak{\frac{x^4+1}{x^4}}+c$
				\item $\brak{\frac{x^4+1}{x^4}}^{\frac{1}{4}}+c$
				\item $\brak{x^4+1}^{\frac{1}{4}}+c$
			\end{enumerate}
		\end{multicols}

10)
	 The integral $\int\frac{2x^{12}+5x^9}{\brak{x^5+x^3+1}^3}dx$ is equal to

		\hfill{(JEE M 2016)}

		\begin{multicols}{2}
			\begin{enumerate}
				\item $\frac{x^5}{2\brak{x^5-x^3+1}^2}+C$
				\item $\frac{-x^{10}}{2\brak{x^5+x^3+1}^2}+C$
				\item $\frac{-x^5}{\brak{x^5+x^3+1}^2}+C$
				\item $\frac{x^{10}}{2\brak{x^5+x^3+1}}+C$
			\end{enumerate}
		\end{multicols}
		where $C$ is an arbitrary constant

11)
	 Let I$_n$ = $\int$tan$^x$dx, \brak{n>1}. I$_4$+I$_6$ = $a \tan^5x + bx^5 + C$, where $C$ is constant of integration, then the ordered pair \brak{a, b} is equal to :

		\hfill{(JEE M 2017)}

		\begin{multicols}{4}
			\begin{enumerate}
				\item$\brak{-\frac{1}{5}, 0}$
					
				\item$\brak{-\frac{1}{5}, 1}$
					
				\item$\brak{\frac{1}{5}, 0}$
					
				\item$\brak{\frac{1}{5}, -1}$
			\end{enumerate}
		\end{multicols}
		
12)
	 The integral $\int\frac{\sin^2x\cos^2x}{\brak{\sin^5x+\cos^3x\sin^2x+\sin^3x\cos^2x+\cos^5x}^2}$dx is equal to

		\hfill{(JEE M 2018)}
		
		\begin{multicols}{2}
			\begin{enumerate}
				\item $\frac{-1}{3\brak{1+\tan ^3x}}$ + $C$
				\item $\frac{1}{1+\cot ^3x}$ + $C$
				\item $\frac{-1}{1+\cot ^3x}$ + $C$
				\item $\frac{1}{3\brak{1+\tan ^3x}}$ + $C$
			\end{enumerate}
		\end{multicols}
		
13)
	 For $x^2\neq$ n$\pi+1$, n$\in$N (the set of natural numbers), the integral $\int x\sqrt{\frac{2\sin \brak{x^2-1}-\sin 2\brak{x^2-1}}{2\sin \brak{x^2-1}+\sin 2\brak{x^2-1}}}$dx is equal to:

		\hfill{(JEE M 2019 - 9 Jan(M))}

		\begin{multicols}{1}
			\begin{enumerate}
				\item $\log_e\abs{\frac{1}{2}\sec ^2\brak{x^2-1}} + c$
				\item $\frac{1}{2}\log_e\abs{\sec ^2\brak{\frac{x^2-1}{2}}} + c$
				\item $\frac{1}{2}\log_e\abs{\sec ^2\brak{\frac{x^2-1}{2}}} + c$
				\item $\log_2\abs{\sec \brak{\frac{x^2-1}{2}}} + c$
			\end{enumerate}
		\end{multicols}
		(where $c$ is a constant of integration)

14)
	 The integral $\int \sec ^{2/3}x\cosec ^{4/3}x dx$ is equal to

		\hfill{(JEE M 2019 - 9 April (M))}

		\begin{multicols}{2}
			\begin{enumerate}
				\item -3$\tan ^{-1/3}x+C$
				\item -$\frac{3}{4}\tan ^{-4/3}x+C$
				\item -3$\cot ^{-1/3}x+C$
				\item 3$\tan ^{-1/3}+C$
			\end{enumerate}
		\end{multicols}
		(Here, $C$ is a constant of integration)

18. Definite Integrals - Section B

31)
	 The area of the region bounded by the parabola $\brak{y-2}^2=x-1$, the tangent of the parabola at the point \brak{2, 3} and the $x$-axis is:

		\hfill{(2009)}

		\begin{multicols}{4}
			\begin{enumerate}
				\item 6
				\item 9
				\item 12
				\item 3
			\end{enumerate}
		\end{multicols}

32)
	 $\int_0^\pi\sbrak{\cot x}dx$, where $\sbrak{.}$ denotes the greatest integer function, is equal to

		\hfill{(2009)}

		\begin{multicols}{4}
			\begin{enumerate}
				\item 1
				\item -1
				\item $-\frac{\pi}{2}$
				\item $\frac{\pi}{2}$
			\end{enumerate}
		\end{multicols}

33)
	 The area bounded between the curves $y=\cos x$ and $y=\sin x$ between the ordinates $x=0$ and $x=\frac{3\pi}{2}$ is

		\hfill{(2010)}

		\begin{multicols}{2}
			\begin{enumerate}
				\item $4\sqrt{2}+2$
				\item $4\sqrt{2}-1$
				\item $4\sqrt{2}+1$
				\item $4\sqrt{2}-2$
			\end{enumerate}
		\end{multicols}

34)
	 Let $p\brak{x}$ be a function defined on \textbf{R} such that $p'\brak{x}=p'\brak{1-x}$, for all $x\in\sbrak{0,1},p\brak{0}=1$ and $p\brak{1}=41$. Then $\int_0^1p\brak{x}dx$ equals
		\hfill{(2010)}

		\begin{multicols}{4}
			\begin{enumerate}
				\item 21
				\item 41
				\item 42
				\item $\sqrt{41}$
			\end{enumerate}
		\end{multicols}

35)
	 The value of $\int_0^1\frac{8\log\brak{1+x}}{1+x^2}dx$ is

		\hfill{(2011)}

		\begin{multicols}{2}
			\begin{enumerate}
				\item $\frac{\pi}{8}\log2$
				\item $\frac{\pi}{2}\log2$
				\item $\log 2$
				\item $\pi \log2$
			\end{enumerate}
		\end{multicols}

36)
	 The area of the region enclosed by the curves $y=x, x=e, y=\frac{1}{x}$ and the positive $x$ axis is

		\hfill{(2011)}

		\begin{multicols}{2}
			\begin{enumerate}
				\item 1 square unit
				\item $\frac{3}{2}$ square units
				\item $\frac{5}{2}$ square units
				\item $\frac{1}{2}$ square unit
			\end{enumerate}
		\end{multicols}

37)
	 The area between the parabolas:$x^2=\frac{y}{4}$ and $x^2=9y$ and the straight line $y=2$ is:
		\hfill{(2012)}

		\begin{multicols}{4}
			\begin{enumerate}
				\item $20\sqrt{2}$
				\item $\frac{10\sqrt{2}}{3}$
				\item $\frac{20\sqrt{2}}{3}$
				\item $10\sqrt{2}$
			\end{enumerate}
		\end{multicols}

38)
	 If $g(x)=\int_0^x\cos 4t dt$, then $g\brak{x+\pi}$ equals

		\hfill{(2012)}

		\begin{multicols}{2}
			\begin{enumerate}[label=(\alph*)]
				\item $\frac{g(x)}{g\brak{\pi}}$
				\item $g(x)+g\brak{\pi}$
				\item $g(x)-g\brak{\pi}$
				\item $g(x).g\brak{\pi}$
			\end{enumerate}
		\end{multicols}

39)
	 \textbf{Statement-1 :} The value of the integral $\int_{\pi/6}^{\pi/3}\frac{dx}{1+\sqrt{\tan x}}$ is equal to $\pi/6$

		\textbf{Statement-2 :} $\int_a^bf\brak{x}dx=\int_a^bf\brak{a+b-x}dx$.

		\hfill{(JEE M 2013)}
		
		(a) Statement-1 is true; Statement-2 is true; Statement-2 is a correct explanation for Statement-1

		(b) Statement-1 is true; Statement-2 is true; Statement-2 is not a correct explanation for Statement-1

		(c) Statement-1 is true; Statement-2 is false

		(d) Statement-1 is false; Statement-2 is true

40)
	 The area (in square units) bounded by the curves $y=\sqrt{x}$, $2y-x+3=0$, $x$-axis, and lying in the first quadrant is :
		\hfill{(JEE M 2013)}

		\begin{multicols}{4}
			\begin{enumerate}
		\item 9
		\item 36
		\item 18
		\item $\frac{27}{4}$
			\end{enumerate}
		\end{multicols}


41)
	 The integral $\int_0^{\pi}\sqrt{1+4\sin ^2\frac{x}{2}-4\sin \frac{x}{2}}dx$ equals:

		\hfill{(JEE M 2014)}

		\begin{multicols}{2}
			\begin{enumerate}
				\item (a) $4\sqrt{3}-4$
				\item (b) $4\sqrt{3}-4-\frac{\pi}{3}$
				\item (c) $\pi-4$
				\item (d) $\frac{2\pi}{3}-4-4\sqrt{3}$
			\end{enumerate}
		\end{multicols}

42)
	 The area of the region described by $A=\{\brak{x,y}:x^2+y^2\leq1$ and $y^2\leq1-x\}$ is:

		\hfill{(JEE M 2014)}

		\begin{multicols}{4}
			\begin{enumerate}
				\item $\frac{\pi}{2}-\frac{2}{3}$
				\item $\frac{\pi}{2}+\frac{2}{3}$
				\item $\frac{\pi}{2}+\frac{4}{3}$
				\item $\frac{\pi}{2}-\frac{4}{3}$
			\end{enumerate}
		\end{multicols}

43)
	 The area (in sq. units) of the region described by $\{\brak{x,y}:y^2\leq2x$ and $y\geq4x-1\}$ is

		\hfill{(JEE M 2015)}

		\begin{multicols}{4}
			\begin{enumerate}
				\item $\frac{15}{64}$
				\item $\frac{9}{32}$
				\item $\frac{7}{32}$
				\item $\frac{5}{64}$
			\end{enumerate}
		\end{multicols}

44)
	The integral $\int_2^4\frac{\log x^2}{\log x^2 + \log\brak{36-12x+x^2}}dx$ is equal to:

		\hfill{(JEE M 2015)}

		\begin{multicols}{4}
			\begin{enumerate}
				\item 1
				\item 6
				\item 2
				\item 4
			\end{enumerate}
		\end{multicols}

45)
	 The area (in sq. units) of the region $\{\brak{x,y}:y^2\geq2x$ and $x^2+y^2\leq4x, x\geq0, y\geq0\}$ is

		\hfill{(JEE M 2016)}

		\begin{multicols}{2}
			\begin{enumerate}
				\item $\pi-\frac{4\sqrt{2}}{3}$
				\item $\frac{\pi}{2}-\frac{2\sqrt{2}}{3}$
				\item $\pi-\frac{4}{3}$
				\item $\pi-\frac{8}{3}$ 
			\end{enumerate}
		\end{multicols} 


\end{document}
