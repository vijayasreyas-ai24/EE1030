\let\negmedspace\undefined
\let\negthickspace\undefined
\documentclass[journal]{IEEEtran}
\usepackage[a5paper, margin=8mm, onecolumn]{geometry}
%\usepackage{lmodern} % Ensure lmodern is loaded for pdflatex
\usepackage{tfrupee} % Include tfrupee package

\setlength{\headheight}{1cm} % Set the height of the header box
\setlength{\headsep}{0mm}     % Set the distance between the header box and the top of the text

\usepackage{gvv-book}
\usepackage{gvv}
\usepackage{cite}
\usepackage{amsmath,amssymb,amsfonts,amsthm}
\usepackage{algorithmic}
\usepackage{graphicx}
\usepackage{textcomp}
\usepackage{xcolor}
\usepackage{txfonts}
\usepackage{listings}
\usepackage{enumitem}
\usepackage{mathtools}
\usepackage{gensymb}
\usepackage{comment}
\usepackage[breaklinks=true]{hyperref}
\usepackage{tkz-euclide} 
\usepackage{listings}
% \usepackage{gvv}                                        
\def\inputGnumericTable{}                                 
\usepackage[latin1]{inputenc}                                
\usepackage{color}                                            
\usepackage{array}                                            
\usepackage{longtable}                                       
\usepackage{calc}                                             
\usepackage{multirow}                                         
\usepackage{hhline}                                           
\usepackage{ifthen}                                           
\usepackage{lscape}
\begin{document}

\bibliographystyle{IEEEtran}
\vspace{3cm}

\title{2023-Jan-29 Shift-2}
\author{AI24BTECH11003 - Badde Vijaya Sreyas}
% \maketitle
% \newpage
% \bigskip
{\let\newpage\relax\maketitle}

\renewcommand{\thefigure}{\theenumi}
\renewcommand{\thetable}{\theenumi}
\setlength{\intextsep}{10pt} % Space between text and floats


\numberwithin{equation}{enumi}
\numberwithin{figure}{enumi}
\renewcommand{\thetable}{\theenumi}

\begin{enumerate}
\setcounter{enumi}{15}
 
%16
    \item If the tangent at a point $P$ on the parabola $y^2 = 3x$ is parallel to the line $x+2y=1$ and the tangents at the points $Q$ and $R$ on the ellipse $\frac{x^2}{4}+\frac{y^2}{1}=1$ are perpendicular to the line $x-y=2$, then the area of the triangle $PQR$ is:

        \begin{multicols}{4}
            \begin{enumerate}
                \item $\frac{9}{\sqrt{5}}$
                \item $5\sqrt{3}$
                \item $\frac{3}{2}\sqrt{5}$
                \item $3\sqrt{5}$
            \end{enumerate}
        \end{multicols}

%17
    \item Let $y=y\brak{x}$ be the solution of the differential equation $x\log_ex\frac{dy}{dx}+y=x^2\log_ex,\brak{x>1}$. If $y\brak{2}=2$, then $y\brak{e}$ is equal to

		\begin{multicols}{4}
			\begin{enumerate}
				\item $\frac{4+e^2}{4}$
				\item $\frac{1+e^2}{4}$
				\item $\frac{2+e^2}{2}$
				\item $\frac{1+e^2}{2}$
			\end{enumerate}
		\end{multicols}

%18
    \item The number of $3$ digit numbers, that are divisible by either $3$ or $4$ but not divisible by $48$, is

        \begin{multicols}{4}
            \begin{enumerate}
                \item $472$
                \item $432$
                \item $507$
                \item $400$
            \end{enumerate}
        \end{multicols}

%19
    \item Let $R$ be a relation defined on $N$ as $a\ R\ b$ is $2a+3b$ is a multiple of 5, $a,b\in N$. Then $R$ is

		\begin{multicols}{2}
			\begin{enumerate}
				\item not reflexive
				\item transitive but not symmetric
				\item symmetric but not transitive
				\item an equivalence relation
			\end{enumerate}
		\end{multicols}

%20
    \item Consider a function $f:N\to R$, satisfying $f\brak{1}+2f\brak{2}+3f\brak{3}+\cdots+xf\brak{x}=x\brak{x+1}f\brak{x};x>2$ with $f\brak{1}=1$. Then $\frac{1}{\brak{2022}}+\frac{1}{f\brak{2028}}$ is equal to 

		\begin{multicols}{4}
			\begin{enumerate}
				\item $8200$
				\item $8000$
				\item $8400$
				\item $8100$
			\end{enumerate}
		\end{multicols}
  
%21
    \item The total number of $4$-digit numbers whose greatest common divisor with $54$ is $2$, is

%22
    \item A triangle is formed by the tangents at the point $\brak{2,2}$ on the curves $y^2=2x$ and $x^2+y^2=4x$, and the line $x+y+2=0$. If $r$ is the radius of its circumcircle, then $r^2$ is equal to
		
%23
    \item A circle with centre $\brak{2,3}$ and radius $4$ intersects the line $x+y=3$ at the points $P$ and $Q$. If the tangents at $P$ and $Q$ intersect at the point $S\brak{\alpha,\beta}$, then $4\alpha-7\beta$ is equal to

%24
    \item Let $a_1=b_1=1$ and $a_n=a_{n-1}++\brak{n-1},b_n=b_{n-1}+a_{n-1}\forall n\geq2$. If $S=\underset{n=1}{\overset{10}{\sum}}\frac{b_n}{2^n}$ and $T=\underset{n=1}{\overset{8}{\sum}}\frac{n}{2^{n-1}}$, then $2^7\brak{2S-T}$ is equal to

%25
    \item If the equation of the normal to the curve $y=\frac{x-a}{\brak{x+b}\brak{x-2}}$ at the point $\brak{1,-3}$ is $x-4y=13$, then the value of $a+b$ is equal to
        
%26
    \item If $A$ be the symmetric matrix such that $\abs{A}=2$ and $\myvec{2 & 1 \\ 2 & \frac{3}{2}}A=\myvec{1 & 2 \\ \alpha & \beta}$. If the sum of the diagonal elements of $A$ is $s$, then $\frac{\beta s}{\alpha^2}$ is equal to 

%27
    \item Let $\left\{ a_k\right\}$ and $\left\{ b_k\right\}, k\in N$, be two G,P,s with common ratio $r_1$ and $r_2$ respectively such that $a_1=b_1=4$ and $r_1<r_2$. Let $c_k=a_k+b_k,k\in N$. If $c_2=5$ and $c_3=\frac{13}{4}$ then $\underset{k=1}{\overset{\infty}{\sum}}c_k-\brak{12a_6+8b_4}$ is equal to
        
%28
    \item Let $X=\left\{ 11,12,13,\cdots,40,41\right\}$ and $Y=\left\{61,62,63,\cdots,90,91\right\}$ be the two sets of observations. If $\overline{x}$ and $\overline{y}$ are their respective means and $\sigma^2$ is the variance  of all the observations in $X\cup Y$, then $\abs{\overline{x}+\overline{y}-\sigma^2}$ is equal to

%29
    \item Let $\alpha =8-14i$, $A=\left\{z\in C:\frac{\alpha z\overline{\alpha}\overline{z}}{z^2-\brak{\overline{z}}^2-112i}=1\right\}$ and $B=\left\{z\in C:\abs{z+3i}=4\right\}$. Then $\underset{z\in A\cap B}{\sum}\brak{Re\ z-Im\ z}$ is equal to

%30
    \item Let $\alpha_1,\alpha_2,\cdots,\alpha_7$ be the roots of the equation $x^7+3x^5-13x^3-15x=0$ and $\abs{\alpha_1}\geq\abs{\alpha_2}\geq\cdots\geq\abs{\alpha_7}$. Then $\alpha_1\alpha_2-\alpha_3\alpha_4+\alpha_5+\alpha_6$ is equal to

\end{enumerate}

\end{document}
