\let\negmedspace\undefined
\let\negthickspace\undefined
\documentclass[journal]{IEEEtran}
\usepackage[a5paper, margin=8mm, onecolumn]{geometry}
%\usepackage{lmodern} % Ensure lmodern is loaded for pdflatex
\usepackage{tfrupee} % Include tfrupee package

\setlength{\headheight}{1cm} % Set the height of the header box
\setlength{\headsep}{0mm}     % Set the distance between the header box and the top of the text

\usepackage{gvv-book}
\usepackage{gvv}
\usepackage{cite}
\usepackage{amsmath,amssymb,amsfonts,amsthm}
\usepackage{algorithmic}
\usepackage{graphicx}
\usepackage{textcomp}
\usepackage{xcolor}
\usepackage{txfonts}
\usepackage{listings}
\usepackage{enumitem}
\usepackage{mathtools}
\usepackage{gensymb}
\usepackage{comment}
\usepackage[breaklinks=true]{hyperref}
\usepackage{tkz-euclide} 
\usepackage{listings}
% \usepackage{gvv}                                        
\def\inputGnumericTable{}                                 
\usepackage[latin1]{inputenc}                                
\usepackage{color}                                            
\usepackage{array}                                            
\usepackage{longtable}                                       
\usepackage{calc}                                             
\usepackage{multirow}                                         
\usepackage{hhline}                                           
\usepackage{ifthen}                                           
\usepackage{lscape}
\begin{document}

\bibliographystyle{IEEEtran}
\vspace{3cm}

\title{2020-Jan-9 Shift-1}
\author{AI24BTECH11003 - Badde Vijaya Sreyas}
% \maketitle
% \newpage
% \bigskip
{\let\newpage\relax\maketitle}

\renewcommand{\thefigure}{\theenumi}
\renewcommand{\thetable}{\theenumi}
\setlength{\intextsep}{10pt} % Space between text and floats


\numberwithin{equation}{enumi}
\numberwithin{figure}{enumi}
\renewcommand{\thetable}{\theenumi}

\begin{enumerate}
\setcounter{enumi}{15}
 
%16
    \item If for all real triplets $\brak{a, b, c}$, $f\brak{x}=a+bx+cx^2$; then $\int_0^1f\brak{x}dx$ is equal to

        \begin{multicols}{1}
            \begin{enumerate}
                \item $2\brak{3f\brak{1}+ 2f\brak{\frac{1}{2}}}$
                \item $\brak{\frac{1}{3}}\brak{f\brak{0}+f\brak{\frac{1}{2}}}$
                \item $\brak{\frac{1}{2}}\brak{f\brak{1}+3f\brak{\frac{1}{2}}}$
                \item ${\frac{1}{6}}\brak{f\brak{0}+f\brak{1}+4f\brak{\frac{1}{2}}}$
            \end{enumerate}
        \end{multicols}

%17
    \item If the number of five digit numbers with distinct digits and 2 at the 10$^{th}$ place is $336k$, then $k$ is equal to:

		\begin{multicols}{4}
			\begin{enumerate}
				\item 8
				\item 7
				\item 4
				\item 6
			\end{enumerate}
		\end{multicols}

%18
    \item Let the observations $x_i\brak{1\leq i \leq 10}$ satisfy the equations,

    $\sum_{i=1}^{10} \brak{x_i - 5}=10$
    and
    $\sum_{i=1}^{10} \brak{x_i-5}^2=40$.
    If $\mu$ and $\lambda$ are the mean and variance of observations, $\brak{x_1-3}, \brak{x_2-3}.....\brak{x_10-3}$, then the ordered pair $\brak{\mu, \lambda}$ is equal to:

        \begin{multicols}{4}
            \begin{enumerate}
                \item $\brak{6, 3}$
                \item $\brak{3, 6}$
                \item $\brak{3, 3}$
                \item $\brak{6, 6}$
            \end{enumerate}
        \end{multicols}

%19
    \item The integral $\int\frac{dx}{\brak{x+4}^{\frac{8}{7}}\brak{x-3}^{\frac{6}{7}}}$ is equal to

		\begin{multicols}{4}
			\begin{enumerate}
				\item $-\brak{\frac{x-3}{x-4}}^{-\frac{1}{7}}+C$
				\item $\frac{1}{2}\brak{\frac{x-3}{x-4}}^{\frac{3}{7}}+C$
				\item $\brak{\frac{x-3}{x-4}}^{\frac{1}{7}}+C$
				\item $-\frac{1}{13}\brak{\frac{x-3}{x-4}}^{-\frac{13}{7}}+C$
			\end{enumerate}
		\end{multicols}

%20
    \item In a box, there are 20 cards out of which 10 are labelled as A and remaining 10 are labelled as B. Cards are drawn at random, one after the other and with replacement, till a second A-card is obtained. The probability that the second A-card appears before the third B-card is:

		\begin{multicols}{4}
			\begin{enumerate}
				\item $\frac{15}{16}$
				\item $\frac{9}{16}$
				\item $\frac{13}{16}$
				\item $\frac{11}{16}$
			\end{enumerate}
		\end{multicols}

%21
    \item If the vectors $\overrightarrow{p} = \brak{a+1}\hat{i}+a\hat{j}+a\hat{k}$, $\overrightarrow{q} = a\hat{i}+\brak{a+1}\hat{j}+a\hat{k}$, and $\overrightarrow{r} = a\hat{i}+a\hat{j}+\brak{a+1}\hat{k}$ $\brak{a\in R}$ are coplanar and $3\brak{\overrightarrow{p}\cdot\overrightarrow{q}}^2 - \lambda\abs{\overrightarrow{r}\times \overrightarrow{q}}^2=0$, then the value of $\lambda$ is

%22
    \item The projection of the line segment joining the points $\brak{1,-1,3}$ and $\brak{2,-4,11}$ on the line joining the points $\brak{-1,2,3}$ and $\brak{3,-2,10}$ is
		
%23
    \item The number of distinct solutions of the equation $\log_{\frac{1}{2}}\abs{\sin x} = 2-\log_{\frac{1}{2}}\abs{\cos x}$ in the interval $\left[0, 2\right]$ is:

%24
    \item If for $x\geq 0$, $y=y\brak{x}$ is the solution of the differential equation $\brak{1+x}dy=\left[ \brak{1+x}^2 +y-3\right]dx$, $y\brak{2}=0$, then $y\brak{3}$ is equal to:
	
%25
    \item The coefficient of $x^4$ in the expansion of $\brak{1+x+x}^{10}$ is

\end{enumerate}

\end{document}
