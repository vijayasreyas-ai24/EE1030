\let\negmedspace\undefined
\let\negthickspace\undefined
\documentclass[journal,9pt,onecolumn]{IEEEtran}
\usepackage[a5paper, margin=8mm, onecolumn]{geometry}
%\usepackage{lmodern} % Ensure lmodern is loaded for pdflatex
\usepackage{tfrupee} % Include tfrupee package

\setlength{\headheight}{1cm} % Set the height of the header box
\setlength{\headsep}{0mm}     % Set the distance between the header box and the top of the text

\usepackage{gvv-book}
\usepackage{gvv}
\usepackage{cite}
\usepackage{amsmath,amssymb,amsfonts,amsthm}
\usepackage{algorithmic}
\usepackage{graphicx}
\usepackage{textcomp}
\usepackage{xcolor}
\usepackage{txfonts}
\usepackage{listings}
\usepackage{enumitem}
\usepackage{mathtools}
\usepackage{gensymb}
\usepackage{comment}
\usepackage[breaklinks=true]{hyperref}
\usepackage{tkz-euclide} 
\usepackage{listings}
% \usepackage{gvv}                                        
\def\inputGnumericTable{}                                 
\usepackage[latin1]{inputenc}                                
\usepackage{color}                                            
\usepackage{array}                                            
\usepackage{longtable}                                       
\usepackage{calc}                                             
\usepackage{multirow}                                         
\usepackage{hhline}                                           
\usepackage{ifthen}                                           
\usepackage{lscape}
\begin{document}

\bibliographystyle{IEEEtran}
\vspace{3cm}

\title{2021-Jul-20 Shift-2}
\author{AI24BTECH11003 - Badde Vijaya Sreyas}
% \maketitle
% \newpage
% \bigskip
{\let\newpage\relax\maketitle}

\renewcommand{\thefigure}{\theenumi}
\renewcommand{\thetable}{\theenumi}
\setlength{\intextsep}{10pt} % Space between text and floats


\numberwithin{equation}{enumi}
\numberwithin{figure}{enumi}
\renewcommand{\thetable}{\theenumi}

\begin{enumerate}
\setcounter{enumi}{0}
 
%1
    \item For the natural numbers $m,n$, if $\brak{1-y}^m\brak{1+y}^n=1+a_1y+a_2y^2+\cdots+a_{m+n}y^{m+n}$ and $a_1=a_2=10$, then the value of $\brak{m + n}$ is equal to

        \begin{multicols}{4}
            \begin{enumerate}
                \item $88$
                \item $664$
                \item $100$
                \item $80$
            \end{enumerate}
        \end{multicols}

%2
    \item The value of $\tan\brak{2\arctan\brak{\frac{3}{5}}+\arcsin\brak{\frac{5}{13}}}$ is equal to

        \begin{multicols}{2}
            \begin{enumerate}
                \item $\frac{-181}{69}$
                \item $\frac{220}{21}$
                \item $\frac{-291}{76}$
                \item $\frac{151}{63}$
            \end{enumerate}
        \end{multicols}

%3
    \item Let $r_1$ and $r_2$ be the radii of the largest and smallest circles, respectively, which pass through the point $\brak{-4,1}$ and having their centres on the circumference of the circle $x^2 +y^2+2x+4y-4=0$. If $\frac{r_1}{r_2}=a+b\sqrt{2}$, then $a+b$ is equal to:

        \begin{multicols}{4}
            \begin{enumerate}
                \item $3$
                \item $11$
                \item $5$
                \item $7$
            \end{enumerate}
        \end{multicols}

%4
    \item Consider the following three statements:\\
    \brak{A}: If $2+4=7$, then $3+4=8$\\
    \brak{B}: If $3+5=8$, then the earth is flat\\
    \brak{C}: If \brak{A} and \brak{B} are true, then $5+6=17$\\
    Then which of the following statements is correct?

		\begin{multicols}{2}
			\begin{enumerate}
				\item \brak{A} is false but \brak{B} and \brak{C} are true
                    \item \brak{A} and \brak{C} are true while \brak{B} is false
                    \item \brak{A} is true while \brak{B} and \brak{C} are false
                    \item \brak{A} and \brak{B} are false while \brak{C} is true
			\end{enumerate}
		\end{multicols}

%5
    \item The lines $x=ay-1=z-2$ and $x=3y-2=bz-2$, $\brak{ab\neq 0}$ are coplanar, if:

		\begin{multicols}{2}
			\begin{enumerate}
				\item $b=1, a\in R-\{0\}$
				\item $a=1, b\in R-\{0\}$
				\item $a=2,b=2$
				\item $a=2,b=3$
			\end{enumerate}
		\end{multicols}

%6
    \item If $\left[x\right]$ denotes the greatest integer less than or equal to $x$, then the value of the integral $\int_{\frac{-\pi}{2}}^{\frac{\pi}{2}}\left[\left[x\right]-\sin x\right]dx$ is equal to:
    
        \begin{multicols}{4}
            \begin{enumerate}
                \item $-\pi$
                \item $\pi$
                \item $-$
                \item $1$
            \end{enumerate}
        \end{multicols}

%7
    \item If the real part of the complex number $\brak{1-\cos\theta + 2i\sin\theta}^{-1}$ is $\frac{1}{5}$ for $\theta\in\brak{0,\pi}$, then the value of the integral $\int_0^\theta\sin x\ dx$ is equal to:

        \begin{multicols}{4}
            \begin{enumerate}
                \item $1$
                \item $2$
                \item $-1$
                \item $0$
            \end{enumerate}
        \end{multicols}
		
%8
    \item Let $f:R-\left\{\frac{\alpha}{6}\right\}\rightarrow R$ be defined by $f\brak{x}=\frac{5x+3}{6x-\alpha}$. Then the value of $\alpha$ for which $\brak{f\circ f}\brak{x}=x$, for all $x\in R-\left\{\{\alpha\}{6}\right\}$, is :

        \begin{multicols}{2}
            \begin{enumerate}
                \item No such $\alpha$ exists
                \item $5$
                \item $8$
                \item $6$
            \end{enumerate}
        \end{multicols}
	
%9
    \item If $f:R\rightarrow R$ is given by $f\brak{x}=x+1$, then the value of\\
    $\underset{x\to\infty}{\lim}\frac{1}{n}\left[f\brak{0}+f\brak{\frac{5}{n}}+f\brak{\frac{0}{n}}+\cdots+f\brak{\frac{5\brak{n-1}}{n}}\right]$\\ is:

        \begin{multicols}{4}
            \begin{enumerate}
                \item $\frac{3}{2}$
                \item $\frac{5}{2}$
                \item $\frac{1}{2}$
                \item $\frac{7}{2}$
            \end{enumerate}
        \end{multicols}

%10
    \item Let A, B and C be three events such that the probability that exactly one of A and B occurs is \brak{1-k}, the probability that exactly one of B and C occurs is \brak{1-2k}, the probability that exactly one of C and A occurs is \brak{1-k} and the probability of all A, B and C occur simultaneously is $k^2$, where $0<k<1$. Then the probability that at least one of A, B and C occur is:

        \begin{multicols}{2}
            \begin{enumerate}
                \item greater than $\frac{1}{8}$ but less than $\frac{1}{4}$
                \item greater than $\frac{1}{2}$
                \item greater than $\frac{1}{4}$ but less than $\frac{1}{2}$
                \item exactly equal to $\frac{1}{2}$
            \end{enumerate}
        \end{multicols}

%11
    \item The sum of all the local minimum values of the twice differentiable function $f:R\rightarrow R$ defined by $f\brak{x}=x^3-3x^2-\frac{3f''\brak{x}}{2}+f''\brak{1}$ is:

        \begin{multicols}{4}
            \begin{enumerate}
                \item $-22$
                \item $5$
                \item $-27$
                \item $0$
            \end{enumerate}
        \end{multicols}

%12
    \item Let in a right angled triangle, the smallest angle be $\theta$.If a triangle formed by taking the reciprocal of it's sides is also a right angled triangle, then $\sin \theta$ is equal to:

        \begin{multicols}{4}
            \begin{enumerate}
                \item $\frac{\sqrt{5}+1}{4}$
                \item $\frac{\sqrt{5}-1}{2}$
                \item $\frac{\sqrt{2}-1}{2}$
                \item $\frac{\sqrt{5}-1}{4}$
            \end{enumerate}
        \end{multicols}

%13
    \item Let $y=y\brak{x}$ satisfies the equation $\frac{dy}{dx}-\abs{A}=0$, for all $x>0$, where $A=\myvec{y&\sin x & 1 \\ 0 & -1 & 1 \\ 2 & 0 & \frac{1}{x}}$. If $y\brak{\pi}=\pi+2$, then the value of $y\brak{\frac{\pi}{2}}$ is:

        \begin{multicols}{4}
            \begin{enumerate}
                \item $\frac{\pi}{2}+\frac{4}{\pi}$
                \item $\frac{\pi}{2}-\frac{1}{\pi}$
                \item $\frac{3\pi}{2}-\frac{1}{\pi}$
                \item $\frac{\pi}{2}-\frac{4}{\pi}$
            \end{enumerate}
        \end{multicols}

%14
    \item Consider the line L given by the equation $\frac{x-3}{2}=\frac{y-1}{1}=\frac{z-2}{1}$. Let Q be the mirror the image of the point $\brak{2,3,-1}$ with respect to L. Let a plane P be such that it passes through Q, and the line L is perpendicular to P. Then which of the following points is on the plane P?

    \begin{multicols}{2}
        \begin{enumerate}
            \item $\brak{-1, 1, 2}$
            \item $\brak{1, 1, 1}$
            \item $\brak{1, 1, 2}$
            \item $\brak{1, 2, 2}$
        \end{enumerate}
    \end{multicols}

%15
    \item If the mean and variance of six observations 7, 10, 11, 15, $a$, $b$ are $10$ and $\frac{20}{3}$, respectively, then the value of $\abs{a-b}$ is equal to:

    \begin{multicols}{4}
        \begin{enumerate}
            \item $9$
            \item $11$
            \item $7$
            \item $1$
        \end{enumerate}
    \end{multicols}
        
\end{enumerate}

\end{document}
